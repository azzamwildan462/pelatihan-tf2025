% pelatihan_rtabmap.tex
\documentclass[12pt,a4paper]{article}
\usepackage[utf8]{inputenc}
\usepackage[T1]{fontenc}
\usepackage[bahasai]{babel}
\usepackage{geometry}
\usepackage{hyperref}
\usepackage{bookmark}
\usepackage{graphicx}
\usepackage{titlesec}
\usepackage{enumitem}
\usepackage{longtable}
\usepackage{listings}
\usepackage{xcolor}
\usepackage{fancyhdr}
\usepackage{float}
\usepackage{caption}
\geometry{margin=2.2cm}

% Header/footer
\pagestyle{fancy}
\fancyhf{}
\fancyhead[L]{Pelatihan ROS 2 — RTAB-Map \& Navigation}
\fancyhead[R]{TurtleBot3, ROS 2 Humble}
\fancyfoot[C]{\thepage}

% code listing style
\lstset{
  basicstyle=\ttfamily\small,
  breaklines=true,
  frame=single,
  captionpos=b,
  columns=fullflexible,
  numbers=left,
  numberstyle=\tiny,
  keywordstyle=\color{blue}\bfseries,
  commentstyle=\color{green!50!black},
  stringstyle=\color{orange}
}

% Title
\title{\Huge\bfseries Modul Pelatihan\\ROS 2: Mapping \& Navigation dengan RTAB-Map}
\author{Disusun oleh: Gemini (Asisten AI) \\ Untuk: Pelatihan Robotika (TurtleBot3)}
\date{22 September 2025}

\begin{document}
\maketitle
\tableofcontents
\newpage

\section*{Ringkasan}
Modul ini dirancang untuk siswa yang sudah paham dasar ROS 2 dan ingin mendalami navigasi otonom, pemetaan dan lokalisasi menggunakan \textbf{RTAB-Map} pada platform \textbf{TurtleBot3 (simulasi Gazebo)}. Setiap modul berisi teori singkat, langkah-langkah praktis, contoh perintah/launch, latihan, dan penilaian.

\section{Learning Outcomes (Hasil Belajar)}
Setelah menyelesaikan modul ini, peserta akan mampu:
\begin{itemize}
    \item Memahami konsep SLAM dan cara kerja RTAB-Map.
    \item Menghasilkan peta lingkungan simulasi dan menyimpannya.
    \item Melakukan lokalisasi terhadap peta yang tersimpan.
    \item Mengonfigurasi dan menjalankan Navigation2 untuk mencapai goal.
    \item Membuat node waypoint untuk misi berurut.
    \item Mendesain state machine sederhana untuk kontrol misi.
    \item Mendeploy (secara simulasi/nyata) program misi terpadu.
\end{itemize}

\section{Overview Modul dan Durasi}
\begin{longtable}{p{3.5cm} p{10cm}}
    Module                             & Durasi (saran) \\
    \hline
    Introduction \& Setup              & 2 jam          \\
    Mapping with RTAB-Map              & 3 jam          \\
    Localization with RTAB-Map         & 2 jam          \\
    Navigation with Nav2               & 3 jam          \\
    Waypoint Navigation                & 2 jam          \\
    State Machines for Mission Control & 3 jam          \\
    Advanced Topics \& Deployment      & 2 jam          \\
    Final Project                      & 6--8 jam (tim) \\
\end{longtable}

\newpage
\section{Module 1: Introduction \& Setup}
\subsection*{Tujuan}
\begin{itemize}
    \item Menyiapkan environment ROS 2 Humble dan TurtleBot3 (simulasi).
    \item Menjalankan Gazebo dan RViz, verifikasi topik sensor.
\end{itemize}

\subsection*{Langkah Praktis (singkat)}
\begin{enumerate}
    \item Buat workspace:
          \begin{lstlisting}[language=bash]
mkdir -p ~/ros2_ws/src
cd ~/ros2_ws
colcon build
source /opt/ros/humble/setup.bash
source install/setup.bash
\end{lstlisting}
    \item Install paket TurtleBot3 \& RTAB-Map (contoh):
          \begin{lstlisting}[language=bash]
sudo apt update
sudo apt install ros-humble-turtlebot3-gazebo \
  ros-humble-rtabmap-ros ros-humble-nav2-bringup
\end{lstlisting}
    \item Jalankan world simulasi:
          \begin{lstlisting}[language=bash]
export TURTLEBOT3_MODEL=burger
ros2 launch turtlebot3_gazebo turtlebot3_world.launch.py
\end{lstlisting}
    \item Cek topics:
          \begin{lstlisting}[language=bash]
ros2 topic list
ros2 topic echo /scan
ros2 topic echo /odom
\end{lstlisting}
\end{enumerate}

\subsection*{Latihan}
\begin{itemize}
    \item Jalankan robot di Gazebo, gunakan teleop untuk menggerakkan dan amati topik sensor di terminal.
\end{itemize}

\newpage
\section{Module 2: Mapping with RTAB-Map}
\subsection*{Tujuan}
\begin{itemize}
    \item Pahami RTAB-Map dan cara membuat peta visual-LiDAR.
    \item Simpan dan muat peta.
\end{itemize}

\subsection*{Langkah Praktis}
\begin{enumerate}
    \item Launch RTAB-Map di simulasi:
          \begin{lstlisting}[language=bash]
# jalankan gazebo world terlebih dahulu
ros2 launch rtabmap_ros rtabmap.launch.py \
  rtabmap_args:="--delete_db_on_start" use_sim_time:=true
\end{lstlisting}
    \item Jelajahi environment dengan teleop sampai peta terbentuk.
    \item Simpan database / peta:
          \begin{lstlisting}[language=bash]
ros2 run rtabmap_ros rtabmap-databaseViewer <db_filename>.db
# atau gunakan service/CLI khusus rtabmap untuk save map (tergantung versi)
\end{lstlisting}
    \item Export map ke format YAML (peta untuk Nav2 dapat dibuat dengan map_server jika tersedia).
\end{enumerate}

\subsection*{Latihan}
\begin{itemize}
    \item Bangun peta ruangan lab/simulasi, simpan peta, dan dokumentasikan langkah.
\end{itemize}

\newpage
\section{Module 3: Localization with RTAB-Map}
\subsection*{Tujuan}
\begin{itemize}
    \item Lakukan lokalisasi terhadap peta yang tersimpan menggunakan RTAB-Map.
    \item Pahami TF tree: \texttt{map -> odom -> base_link}.
\end{itemize}

\subsection*{Langkah Praktis}
\begin{enumerate}
    \item Muat peta/DB RTAB-Map lalu jalankan mode localization (lihat dokumentasi rtabmap untuk argumen `--database_path` atau parameter `rtabmap/DatabasePath`).
          \begin{lstlisting}[language=bash]
ros2 launch rtabmap_ros rtabmap.launch.py \
  rtabmap_args:="--database_path /path/to/my_map.db" use_sim_time:=true
\end{lstlisting}
    \item Pastikan TF terlihat di RViz: tambahkan display TF, Pose, LaserScan.
    \item Verifikasi robot tetap terkunci pada pose sesuai peta.
\end{enumerate}

\subsection*{Latihan}
\begin{itemize}
    \item Lakukan \"kidnap test\": pindahkan robot di gazebo manual, cek apakah RTAB-Map bisa re-localize atau perlu re-init.
\end{itemize}

\newpage
\section{Module 4: Navigation with Nav2}
\subsection*{Tujuan}
\begin{itemize}
    \item Mengenal Navigation2 (Nav2): planners, controllers, costmap.
    \item Mengirim goal & observasi path planning.
\end{itemize}

\subsection*{Langkah Praktis}
\begin{enumerate}
    \item Launch Nav2 (contoh):
          \begin{lstlisting}[language=bash]
ros2 launch nav2_bringup navigation_launch.py \
  use_sim_time:=true map:=/path/to/map.yaml
\end{lstlisting}
    \item Kirim goal dari RViz2 (Publish Goal).
    \item Lakukan tuning sederhana pada `costmap_common_params.yaml`:
          \begin{itemize}
              \item `inflation_radius`, `obstacle_range`, `raytrace_range`
              \item `global_planner` vs `local_planner` (contoh: NavFn + DWB)
          \end{itemize}
\end{enumerate}

\subsection*{Latihan}
\begin{itemize}
    \item Kirim beberapa goal, catat kegagalan dan penyebab (collision, planner failure, tf).
\end{itemize}

\newpage
\section{Module 5: Waypoint Navigation}
\subsection*{Tujuan}
Buat node untuk mengirimkan serangkaian goal (waypoints) ke Nav2 secara otomatis.

\subsection*{Contoh Python: node pengirim waypoints (template)}
\begin{lstlisting}[language=python]
#!/usr/bin/env python3
import rclpy
from rclpy.node import Node
from geometry_msgs.msg import PoseStamped
from rclpy.duration import Duration
from rclpy.qos import QoSProfile
from nav2_msgs.action import NavigateToPose
from rclpy.action import ActionClient

class WaypointClient(Node):
    def __init__(self):
        super().__init__('waypoint_client')
        self._action_client = ActionClient(self, NavigateToPose, 'navigate_to_pose')
        # Define waypoints (sample)
        self.waypoints = [
            {'x':1.0,'y':0.0,'yaw':0.0},
            {'x':1.0,'y':1.0,'yaw':1.57},
            {'x':0.0,'y':1.0,'yaw':3.14},
        ]
        self.timer = self.create_timer(1.0, self.run)

    def run(self):
        if not self._action_client.wait_for_server(timeout_sec=2.0):
            self.get_logger().info('Action server not available yet...')
            return
        self.timer.cancel()
        self.send_waypoints()

    def send_waypoints(self):
        for wp in self.waypoints:
            pose = PoseStamped()
            pose.header.frame_id = 'map'
            pose.header.stamp = self.get_clock().now().to_msg()
            pose.pose.position.x = wp['x']
            pose.pose.position.y = wp['y']
            # orientation from yaw (simple)
            from tf_transformations import quaternion_from_euler
            q = quaternion_from_euler(0,0,wp['yaw'])
            pose.pose.orientation.x = q[0]
            pose.pose.orientation.y = q[1]
            pose.pose.orientation.z = q[2]
            pose.pose.orientation.w = q[3]
            # send action request (omitted request building boilerplate for brevity)
            # see nav2_msgs/NavigateToPose action usage
            self.get_logger().info(f'Sending waypoint: {wp}')
        self.get_logger().info('All waypoints sent.')

def main(args=None):
    rclpy.init(args=args)
    node = WaypointClient()
    rclpy.spin(node)
    node.destroy_node()
    rclpy.shutdown()
\end{lstlisting}

\subsection*{Latihan}
\begin{itemize}
    \item Modifikasi waypoint list untuk patroli.
    \item Tambahkan re-try apabila goal gagal tercapai.
\end{itemize}

\newpage
\section{Module 6: State Machines for Mission Control}
\subsection*{Tujuan}
Mengatur alur misi (patrol → detect → return) menggunakan state machine.

\subsection*{Pilihan tool}
\begin{itemize}
    \item \textbf{smach} (umumnya ROS1, tetapi ada adaptasi/alternatif untuk ROS2) — bila ingin state-machine classic.
    \item \textbf{FlexBE} — tool GUI untuk behavior execution (support ROS2 via bridge/porting).
    \item Alternatif sederhana: implementasi state machine manual di Python dengan kelas.
\end{itemize}

\subsection*{Contoh sederhana state machine (manual Python)}
\begin{lstlisting}[language=python]
class MissionStateMachine:
    def __init__(self):
        self.state = 'IDLE'
    def step(self, event=None):
        if self.state == 'IDLE' and event == 'start':
            self.state = 'PATROL'
        elif self.state == 'PATROL' and event == 'target_found':
            self.state = 'APPROACH'
        elif self.state == 'APPROACH' and event == 'task_done':
            self.state = 'RETURN'
        elif self.state == 'RETURN' and event == 'arrived':
            self.state = 'IDLE'
\end{lstlisting}

\subsection*{Latihan}
\begin{itemize}
    \item Gabungkan node waypoint dengan state machine: saat state PATROL, jalankan waypoints; jika sensor mendeteksi objek, transit ke APPROACH lalu eksekusi task.
\end{itemize}

\newpage
\section{Module 7: Advanced Topics \& Deployment}
\subsection*{Topik}
\begin{itemize}
    \item Kidnapped robot: strategi re-localization dan recovery behavior.
    \item Sensor fusion: IMU + encoder odometry + RTAB-Map pose.
    \item Deploying ke TurtleBot fisik: langkah safety, kalibrasi, dan pengecekan.
\end{itemize}

\subsection*{Latihan}
\begin{itemize}
    \item Simulasi kidnapped: hentikan node Nav2, pindahkan robot di Gazebo, jalankan re-localize.
    \item Jika ada TurtleBot fisik: deploy peta dan jalankan misi sederhana dengan supervisi.
\end{itemize}

\newpage
\section{Final Project (Instruksi)}
\subsection*{Deskripsi}
Tim mahasiswa membuat program misi lengkap: build map (RTAB-Map) → lokalisi → patroli waypoint → kontrol misi (state machine) → laporkan hasil.

\subsection*{Deliverables}
\begin{itemize}
    \item Repository Git berisi launch files, node Python/C++, konfigurasi Nav2, peta (.db/.yaml), dan README.
    \item Video demonstrasi simulasi (maks 5 menit).
    \item Laporan 2--4 halaman (PDF) yang merangkum arsitektur, masalah & solusi.
\end{itemize}

\subsection*{Rubrik Penilaian (contoh)}
\begin{itemize}
    \item Sistem bekerja (mapping + localization + waypoint): 40\%
    \item State machine & robustness: 20\%
    \item Kualitas kode & dokumentasi: 20\%
    \item Presentasi & demo: 20\%
\end{itemize}

\newpage
\section{Appendix A: Resources \& Referensi}
\begin{itemize}
    \item RTAB-Map ROS: \url{http://wiki.ros.org/rtabmap_ros}
    \item Navigation2: \url{https://navigation.ros.org/}
    \item TurtleBot3: \url{https://emanual.robotis.com/docs/en/platform/turtlebot3/}
    \item LaTeX tips: gunakan \texttt{latexmk} atau \texttt{xelatex} untuk compile.
\end{itemize}

\section{Appendix B: Checklist Persiapan Lab}
\begin{itemize}
    \item Installasikan ROS 2 Humble pada semua mesin peserta (atau sediakan VM).
    \item Pastikan Gazebo & TurtleBot3 packages terpasang.
    \item Sediakan satu mesin/instruktur yang menjadi host (untuk remote desktop / screen sharing).
    \item Siapkan sample world & peta (opsional).
\end{itemize}

\end{document}
